\thispagestyle{empty}
\chapter{Волновая оптика}
\clearpage
\section{Поляризация}
\begin{problem}
Объяснить, почему в показанной на рисунке установке при приложении к пластинам напряжения, свет начинает беспрепятственно проходить через установку.
\end{problem}
\section{Надо придумать название}
\begin{problem}
Известно, что у большинства веществ, например у стекла, магнитная и электрическая проницаемость положительны. Особый интерес в оптике представляет плазма, так как на частотах меньше собственной\footnote{см. умную книжку} электрическая её проницаемость меньше нуля. Однако, уравнения Максвелла и уравнения связи не теряют смысл и при одновременной отрицательности как магнитной, так и электрической проницаемостей вещества. Начертите картину прохождения лучей света через двояковогнутую линзу из вещества с $\varepsilon<0$ и $\mu<0$ одновременно. Попытайтесь найти другие интересные свойства таких веществ.
\end{problem}