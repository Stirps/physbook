\chapter{Электричество и магнетизм}
\thispagestyle{empty}
\clearpage
\section{Электричество}
\begin{problem}
Имеется источник стационарного электрического поля. Как нужно расположить лампу накаливания и какова должна быть величина поля $\vec{E}$, чтобы лампа светилась? Известны: длина спирали $l$, удельное сопротивление $\rho$, теплоёмкость всей нити в целом $c$ и температура $t$, при которой нить интенсивно светится.
\end{problem}
\begin{problem}
Имеется источник высокого напряжения, к положительному полюсу которого подключена игла, закреплённая над установкой, а к отрицательному полюсу - сложенная в виде треугольника проволочка. К проволочке, в вершинах, приделаны диэлектрические стойки, соединяющие её и треугольник из алюминиевой фольги (так же за вершины). После включения источника, через некоторое время (зависит от напряжения на нём), треугольный электрод взлетает на некоторую высоту. Объяснить это явление.
\end{problem}
\begin{problem}
Обычно тонкая струйка воды внизу дробится на капли. Однако, если поднести к струйке заряженный предмет, она до конца останется сплошной. Как это объяснить?
\footnote{Прежде всего нужно объяснить, почему происходит дробление струйки на капли.}
\end{problem}