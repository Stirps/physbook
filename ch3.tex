\chapter{Механика}
\thispagestyle{empty}
\clearpage
\begin{problem}
C наклонной плоскости с высоты $h$ соскальзывает брусок. В начальной точке он покоится и его энергия равна потенциальной: $\left.E_p\right|_{t=0}=mgh$. В конечной точке брусок приобретает горизонтальную скорость $v$, соответственно его энергия равна кинетической: $\left.E_k\right|_{t}=\frac{mv^2}{2}$, потенциальная энергия равна нулю. Перейдем в систему отсчета, движущуюся относительно первой со скоростью v (скорость бруска в конечной точке пути). Тогда в начальной точке его энергия равна сумме кинетической энергии и потенциальной $\left.E_p\right|_{t=0}=\frac{mv^2}{2}+mgh$. В конечной точке и кинетическая и потенциальная энергии обращаются в ноль. Куда делась энергия?
\end{problem}